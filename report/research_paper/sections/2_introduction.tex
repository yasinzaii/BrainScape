Magnetic resonance imaging (MRI) has revolutionized our understanding of the human brain by providing 
non-invasive insights into both its macroscale organization and its microstructural properties. From 
macro-level assessments using diffusion tractography and functional connectivity analyses to more 
fine-grained techniques like quantitative T1 (qT1) relaxation mapping, MRI offers a versatile window 
into the architecture and function of neural systems (\cite{sereno2013mapping}, \cite{park2013structural},
\cite{honey2009predicting}).
For instance, qT1 mapping can effectively distinguish heavily myelinated from lightly myelinated cortical 
regions, consistent with early post-mortem myeloarchitectonic findings (\cite{nieuwenhuys2013myeloarchitectonic, shams2019comparison}), 
and thereby enables in vivo profiling of cortical microstructure. This non-invasive approach offers a 
significant advantage by permitting direct comparison between histological myeloarchitecture and in vivo 
microstructural measures.
Moreover, when qT1-based metrics are integrated with anatomical distance measures (e.g., geodesic 
distances along the cortical surface or wiring cost principles), these techniques provide new 
insights into how local cortical regions are organized within large-scale brain networks, 
thereby advancing our comprehension of the brain's structural and functional architecture.
\cite{hong2018multidimensional, ecker2013intrinsic, paquola2020multi}.

Neuroscience research has significantly advanced through the adoption of open science practices, especially 
with the sharing of open data (\cite{snoek2021amsterdam, van2013wu, miller2016multimodal, di2014autism}).
Despite significant advancements in MRI acquisition and analysis, neuroscience research continues to face 
fundamental obstacles in gathering, curating, and harmonizing neuroimaging datasets for robust, 
data-driven discoveries. 

The vast majority of shared datasets are small, are often distributed across scattered repositories 
and stored in incompatible formats. This fragmentation hinders the efforts of data pooling, for curating enriched
multicentric datasets (\cite{dishner2024survey}). On the other hand, there are also numerous and 
widely used data sharing initiatives for multimodal MRI data, such as the Human Connectome Project
(\cite{van2013wu}), UK BioBank (\cite{miller2016multimodal}), CORR (\cite{gorgolewski2017preprocessed}), ABIDE 
(\cite{di2014autism}), FC1000 (\cite{biswal2010toward}). However, access to some of these datasets 
requires fees or submission of an application, and the majority of these datasets have extensive 
user agreements, which makes it challenging for an average researcher to access and further pool them.
Furthermore, the collection of new MRI data is hampered by ethical and logistical complexities, 
including the necessity for institutional review board approvals, patient privacy protections, 
and significant imaging costs, all of which limit the volume of data that can be readily assembled
(\cite{white2022data, sardanelli2018share}).
Additionally, variability across multiple sites and scanners introduces biases that can significantly 
impact downstream analyses, such as voxel-based morphometry and diffusion metrics 
(\cite{takao2014effects}), lesion volumes (\cite{shinohara2017volumetric}), and DTI measurements (\cite{zhu2011quantification}). 
Compounding this issue is the absence of standardized metadata on demographics and clinical variables, 
which complicates the merging of smaller specialized collections in a reproducible manner (\cite{pomponio2019harmonization}).

On the other hand, the demand for machine learning and artificial intelligence applications in 
neuroimaging has rapidly increased. State-of-the-art deep learning methods require large, diverse, 
and well-annotated datasets to effectively train models capable of generalizing across populations, 
scanners, and clinical conditions (\cite{dishner2024survey}). However, data fragmentation, data access limitations 
restrictive licensing, regulatory barriers, and inconsistent annotation protocols often hinder achieving the necessary 
scale and pooling of the different datasets (\cite{goldfarb2022ai}). Furthermore, integrating data from multiple sources, 
each with varying acquisition parameters, demographic coverage, and consent restrictions, remains a significant challenge 
(\cite{pomponio2019harmonization}).

Several open neuroimaging platforms, including OpenNeuro and the NeuroImaging Tools and Resources Collaboratory (NITRC),
have emerged to facilitate data sharing and standardization (\cite{markiewicz2021openneuro, buccigrossi2008neuroimaging}).
OpenNeuro (\url{https://openneuro.org/}) has become one of the largest repositories, hosting more than 1200 public 
datasets with free access and most datasets under Creative Commons Zero licensing. This makes it an ideal choice for 
pooling datasets to increase their size and heterogeneity. 
Furthermore, OpenNeuro utilizes the robust Brain Imaging Data Structure (BIDS) format, which standardizes data 
organization and supports reproducible workflows, thereby reducing manual labor in data handling (\cite{markiewicz2021openneuro}). 
Researchers aiming to combine smaller, specialized repositories with larger, well-established databases often must 
resort to ad hoc scripts to reconcile directory structures, manage irregular metadata, and ensure consistent preprocessing 
steps, such as skull stripping, normalization, or anatomical registration.


To address these practical hurdles and accelerate data-driven investigations, we propose an open-source Python framework 
that streamlines the collection, integration, and preprocessing of anatomical MRI data from \NumDatasets\ different datasets. 
This framework automates typical data handling processes and enforces standardized methods to ensure reproducibility.
Moreover, this framework allows dynamic customization, enabling researchers to incorporate new datasets, demographic information, 
and specialized preprocessing modules without the overhead of reworking an entire codebase.
This framework is built upon a plugin-based architecture, designed to:


\begin{itemize}
    
    \item \textbf{Automate Data Handling:} 
    Automatically download raw MRI scans, regardless of format, and map them into a standardized structure via a mapping plugin. 
    This hands-free workflow minimizes manual intervention and reduces human error.

    \item \textbf{Incorporate Demographic Metadata:}
    Employ a YAML-based schema to attach relevant demographic fields (e.g., age, sex, race, handedness, clinical status) to each subject, 
    enabling more comprehensive, population-specific analyses.

    \item \textbf{Promote Reproducibility:}
    Enforce transparent and traceable data processing pipelines through configurable default and data specific parameters, ensuring 
    that the all operations are traceable and same operations are applied uniformly across multiple datasets.

    \item \textbf{Facilitate Extensibility:}
    A modular design enables researchers to easily integrate new data sources, demographic fields, or specialized preprocessing workflows 
    without overhauling the existing codebase. The configurations for each dataset are saved in individual configuration files, allowing 
    for tailored updates and corrections specific to each dataset.

\end{itemize}




By consolidating multiple open datasets and enforcing standardized practices, the framework 
reduces fragmentation and supports large-scale MRI-based research. Its automated procedures 
alleviate the burden of repetitive tasks, freeing researchers to focus on higher-level analyses 
such as structural biomarker discovery, group-level comparisons of clinical populations, 
or advanced machine learning modeling.
Furthermore, we also have kept track licensing and usage permissions of every individual dataset, 
to ensures that ethical standards are upheld and proprietary data is respected when aggregating 
resources from various repositories.
In this paper, we detail the architecture, functionalities of this framework.
% We also present several use cases illustrating how the platform simplifies the process of acquiring, preprocessing, and assembling anatomical MRI scans from diverse sources. 
Ultimately, we aim to empower the neuroscience community to build more inclusive, large-scale datasets 
that fuel robust computational models and advance our knowledge of brain structure and function.

