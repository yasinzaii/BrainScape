Magnetic resonance imaging (MRI) has revolutionized our understanding of brain structure and function,
enabling non-invasive studies of brain development, plasticity, and pathology in healthy and clinical cohorts (\cite{katti2011magnetic, mills2014methods}).
%MRI is a non-invasive imaging technique that uses non-ionizing electromagnetic radiation to generate high-resolution, 
%three-dimensional images of internal brain structures.
%This imaging technique is versatile as it allows the customization of acquisition and analysis protocols to emphasize specific anatomical features, 
%such as white matter tracts, lesions, and arteries (\cite{katti2011magnetic}).
%Unlike historical reliance on post-mortem histology, MRI offers a safe and non-invasive way to explore both healthy and diseased brains.
%Its versatility in clinical diagnostics enables the identification of structural abnormalities such as lesions and atrophy, facilitating early intervention in neurodegenerative diseases.
MRI research has advanced through the adoption of open science practices, especially 
with the open sharing of MRI data (\cite{snoek2021amsterdam, van2013wu, miller2016multimodal, di2014autism}).
Large-scale initiatives are valuable resources for population-level analyses 
and have contributed to big data efforts in neuroimaging, with some examples including the 
Human Connectome Project (HCP; \cite{van2013wu}), CORR (\cite{gorgolewski2017preprocessed}), 
FC1000 (\cite{biswal2010toward}), ABCD (\cite{casey2018adolescent}), GSP (\cite{holmes2015brain}), ADNI (\cite{mueller2005alzheimer}), 
and UK Biobank (\cite{miller2016multimodal}) 
%However, small, specialized datasets are crucial for capturing diseases and specific phenotypes, such as schizophrenia, 
% epilepsy and prosopagnosia (\cite{soler2022brain, taylor2024imaging, noad2024familiarity} ).
% While large-scale consortia enable the detection of subtle yet meaningful associations at the population level, 
% smaller, specialized studies are particularly effective in examining targeted hypotheses through controlled experimental designs, 
% within-subject manipulations, or in clinical populations (\cite{gratton2022brain}).
Thanks to the open science ethos within the neuroimaging community, thousands of MRI datasets are now publicly available, 
facilitating research across various aspects of health and disease. 
Large-scale consortia are valued for their consistent acquisition protocols, 
enriched demographic details, and broad population coverage, 
while, smaller specialized datasets remain critical for studying neurological and psychiatric disorders 
(\cite{soler2022brain, fajardo2024functional, gibson2024aphasia}). 
%These smaller datasets can be difficult to aggregate into large-scale analyses due to challenges with data fragmentation and organisation, and are often underutilised.


Large neuroimaging datasets require substantial time and resource burden. 
For example, \cite{horien2021hitchhiker} found that two or three researchers typically 
require about six to nine months to download and process raw data from thousands of participants. 
Without an automated framework, researchers often invest considerable effort in downloading, 
ad hoc organizing, preprocessing, and performing quality control. 
This manual process can also tie investigators to previously preprocessed pipelines, 
restricting the scope of subsequent studies, as reprocessing large datasets again 
requires enormous effort (\cite{horien2021hitchhiker}). 
Additionally, careful organization and comprehensive documentation of every processing step is essential for reproducibility. 
By adopting widely accepted standards like the Brain Imaging Data Structure (BIDS) and 
maintaining detailed records of software versions, code, and workflow decisions, 
researchers can ensure that other investigators can precisely replicate their study pipeline (\cite{white2022data, horien2021hitchhiker}). 
% Furthermore, before sharing any derivative datasets, investigators must check data usage agreements (DUAs) of the original dataset 
% to ensure compliance with the data-sharing terms and consult their local institutional review board (IRB) or human investigation committee (HIC) 
% to uphold ethical and legal guidelines (\cite{horien2021hitchhiker}).

%While there are numerous large-scale multimodal MRI data-sharing initiatives, 
%with richly characterized cohorts such as UK Biobank, ABCD, ADNI, GSP, 
%these resources may require access fees or administrative approvals 
%(\cite{miller2016multimodal, casey2018adolescent, mueller2005alzheimer,   holmes2015brain}). 
Large-scale consortia provide extensive sample sizes, offering the statistical 
power necessary to ensure the replicability of research findings, 
thereby effectively reducing false positives and minimizing inflated effect 
sizes in studies of brain-behavior associations and phenotype predictions (\cite{marek2024replicability}). 
While reproducibility establishes an essential baseline for validating findings, 
it alone does not ensure clinical utility or true generalizability (\cite{kiar2024experimental}). 
Furthermore, these large datasets may lack generalizability due to demographic biases 
and over-representing specific populations, limiting their applicability in broader clinical 
and real-world contexts (\cite{marek2024replicability, yang2024limits}).
In addition, many large consortium neuroimaging initiatives intentionally 
minimize dataset variability, creating harmonization that 
does not exist in real-world scenarios (\cite{adkinson2024brain}). 
Moreover some large-scale MRI data-sharing initiatives such as UK Biobank, ABCD, ADNI, and GSP 
may require access fees or administrative approvals (\cite{miller2016multimodal, casey2018adolescent, mueller2005alzheimer,   holmes2015brain}). 
These policies are intended to ensure responsible data use, 
protect participant confidentiality, and uphold ethical standards (\cite{white2022data}). 
However, while they help safeguard personal information, they may also create accessibility barriers 
for researchers with limited administrative support or funding.
%Although these large-scale datasets provide richly characterized cohorts, 
%they may have inherent sampling biases (particularly when focusing on normative data), and may 
%often overlook diverse clinical populations and rare phenotypes. 
Another concern is data decay, which occurs when the repeated analysis 
of the same high-profile datasets leads to overfitting and reduces 
the generalizability of findings over time (\cite{thompson2020dataset, horien2021hitchhiker}).

Meanwhile, thousands of smaller, specialized, and openly accessible datasets 
provide an alternate source of MRI data for neuroscientists 
who may not otherwise have the resources to acquire data or access data from large-scale repositories. 
These MRI datasets are typically defaced and exclude sensitive demographic or genetic details, 
thereby minimizing privacy concerns.
While they generally lack the richly characterized cohorts found in many of the large-scale repositories, 
such smaller datasets are still valuable in augmenting larger datasets for data-driven analyses. 
%Consequently, there is often a trade-off between the depth of available data versus the ease of access.
These smaller, specialized ``boutique'' datasets usually focus on specific brain disorders and targeted research studies. 
A promising way forward lies in open-science collaborations that combine diverse, 
heterogeneous datasets, enhancing both sample diversity and statistical power, 
and thereby improving reproducibility and generalizability 
of predictive brain-behavior associations (\cite{marek2024replicability, adkinson2024brain, yang2024limits}).
%Pooling these datasets with large-scale consortia can enhance the dataset's clinical relevance, 
%while enabling more advanced, data-driven analyses of diverse neurological conditions. 
%Moreover, integrating these smaller datasets with widely used large-scale datasets helps mitigate data decay, 
%prevent overfitting, and improve the generalizability of research analyses (\cite{horien2021hitchhiker}).
Pooling these specialized datasets with large-scale consortia can broaden clinical relevance, 
mitigate data decay, prevent overfitting, and improve generalizability, 
thus enabling more advanced, data-driven analyses for diverse neurological conditions
(\cite{horien2021hitchhiker, marek2024replicability, adkinson2024brain, yang2024limits}).
Both \cite{kiar2024experimental} and \cite{adkinson2024brain} emphasize the importance 
of embracing dataset variability, including differences in demographics, 
scanner hardware, clinical status, and acquisition protocols, to build more 
robust and transferable models. 

However, these resources are dispersed across multiple repositories and stored in incompatible formats, 
leading to fragmentation that complicates data pooling, for curating enriched multicentric datasets (\cite{dishner2024survey}). 
%In addition, the collection of new MRI data requires strict adherence to ethical guidelines, the safeguarding of participant privacy, 
%and the management of substantial imaging costs, all of which limit the volume of data that can be readily assembled (\cite{white2022data, sardanelli2018share}).
Furthermore, variability across multiple sites and scanners introduces biases that can significantly 
impact downstream analyses. 
%such as voxel-based morphometry (\cite{takao2014effects}), diffusion metrics (\cite{zhu2011quantification}), 
%and lesion volumes (\cite{shinohara2017volumetric}). 
Reducing multi-site scanner variability in MRI datasets is an active area of research, 
with harmonization techniques like ComBat increasingly employed to mitigate these differences 
and enhance the reliability of multi-site studies (\cite{fortin2017harmonization, fortin2018harmonization}).
Although harmonization techniques are increasingly employed to remove these differences, 
they can also remove real-world variability, which is often present in clinical settings. 
\cite{adkinson2024brain}, for example, demonstrated that unharmonized developmental samples 
can produce robust cross-dataset predictions, occasionally outperforming models trained 
within a single harmonized dataset. 
Compounding these challenges is the lack of standardized metadata on demographics and clinical variables, 
which complicates the merging of smaller specialized datasets in a reproducible manner (\cite{pomponio2019harmonization}).



On the other hand, the demand for machine learning and artificial intelligence applications in neuroimaging has rapidly increased. 
Advanced deep learning models require large, diverse, and well-annotated datasets to effectively train models capable of generalizing across populations, 
scanners, and clinical conditions (\cite{dishner2024survey}). 
Models trained on narrowly selected samples risk shortcut learning, 
where the model captures associations between brain and unintended confounding 
variables rather than underlying brain-behavior relationships (\cite{marek2024replicability, yang2024limits}). 
Additionally, distribution shifts across populations and imaging protocols can undermine an AI model's 
fairness and generalizability, particularly when models rely on demographic shortcuts, 
leading to biased and unreliable predictions (\cite{yang2024limits}). 
Pooling data from multiple datasets is a promising strategy to meet these requirements, 
enabling researchers to leverage cutting-edge artificial intelligence and deep learning approaches with enhanced generalizability.
However, assembling and integrating such datasets remains challenging due to data fragmentation, 
limited data access, inconsistent annotation protocols, heterogeneous acquisition parameters, 
and demographic variability (\cite{goldfarb2022ai, pomponio2019harmonization}).

Several open neuroimaging platforms, including OpenNeuro and the NeuroImaging Tools and Resources Collaboratory (NITRC),
have emerged to facilitate data sharing and standardization (\cite{markiewicz2021openneuro, buccigrossi2008neuroimaging}).
OpenNeuro (\url{https://openneuro.org/}) has become one of the largest repositories, hosting more than 1200 public 
datasets with free access and most datasets under Creative Commons Zero (CC0) licensing. This makes it an ideal choice for 
pooling datasets to increase their size and heterogeneity. 
Furthermore, OpenNeuro utilizes the robust Brain Imaging Data Structure (BIDS) format, which standardizes data 
organization and supports reproducible workflows, thereby reducing manual labor in data handling (\cite{markiewicz2021openneuro}). 
Researchers aiming to collate these smaller, specialized repositories often must 
resort to ad hoc scripts to reconcile directory structures, manage irregular metadata, and ensure consistent preprocessing 
steps, such as skull stripping, normalization, or anatomical registration.


To overcome these challenges and enhance data-driven research, we introduce BrainScape,
which consists of two linked components: 
(i) the \emph{BrainScape framework}, an open-source, plugin-based Python workflow, and 
(ii) the \emph{BrainScape dataset}, a curated collection of \NumDatasets\ publicly available MRI repositories distributed as dataset-specific configuration files.
Each dataset is represented by a comprehensive configuration file that specifies precise instructions 
required by the framework for downloading, organizing, preprocessing, and demographic annotation. 
By integrating this flexible software framework with dataset-specific configurations, 
BrainScape enables researchers to reproduce analysis-ready datasets through a consistent and transparent workflow.
Our approach directly addresses the objective of improving reproducibility and 
generalizability highlighted by \cite{marek2024replicability}, \cite{yang2024limits}, \cite{kiar2024experimental}, and \cite{adkinson2024brain}.
%BrainScape offers a scalable solution to manage the variability and heterogeneity found in both healthy and clinical cohorts. 
BrainScape framework allows combining smaller, specialized datasets with large-scale collations, 
allowing generalizing across diverse populations as well as clinical cohorts. 
BrainScape maintains the integrity of each source dataset's structure and metadata, ensuring no original information is lost or any bias is introduced from duplicate images. 
It keeps every MRI image linked to its original source, retains a record of excluded subjects, 
and enforces standardized methods to ensure reproducibility. 
BrainScape significantly reduces the typical workload associated with downloading, 
organizing, and preprocessing MRI data from thousands of participants, 
by automating the entire workflow for integrating and preprocessing datasets. 
BrainScape framework offers researchers a rapid start while still allowing complete workflow control.
Moreover, the software framework allows dynamic customization, enabling researchers to incorporate new datasets, demographic information, 
and specialized preprocessing modules without the overhead of reworking the codebase.
Researchers can share their BrainScape workflow configurations and associated plugins, 
allowing others to replicate and recreate preprocessed images locally. 
This approach avoids redistributing derivative datasets, 
while ensuring reproducible pipelines across different research groups. 
BrainScape framework is built upon a plugin-based architecture, designed to:

\begin{itemize}
    
    \item \textbf{Automate data handling:} 
    Automatically download raw MRI scans, regardless of format, and map them into a standardized structure via a mapping plugin. 
    This hands-free workflow minimizes manual intervention and reduces human error.

    \item \textbf{Demographic metadata integration:}
    Employ a YAML-based schema to attach relevant demographic fields (e.g., age, sex, race, handedness, clinical status) to each subject, 
    enabling comprehensive, population-specific analyses.

    \item \textbf{Reproducibility:}
    Enforce transparent and traceable data processing pipelines through configurable default and dataset-specific parameters, ensuring 
    that all operations are traceable and the same operations are applied uniformly across multiple datasets.

    \item \textbf{Extensibility:}
    Maintain a modular design that allows the integration of new data sources, demographic fields, and specialized workflows for data handling and preprocessing 
    without overhauling the existing codebase. 
    Each dataset's configuration is stored separately, allowing for tailored updates and corrections.

\end{itemize}


By consolidating multiple open datasets and enforcing standardized methods, BrainScape reduces fragmentation, 
supports large-scale MRI research, and automates repetitive data-handling tasks. 
This enables researchers to focus on higher-level analyses such as structural biomarker discovery, 
or machine learning applications.
Furthermore, we have kept track of the licensing and usage permissions of every dataset to ensure that ethical 
standards are upheld and proprietary data is respected when aggregating resources from various repositories.
To properly credit each source repository included in BrainScape's dataset, we maintain a dedicated ``README'' file 
that identifies each dataset's source, cites relevant publications, and provides direct links to original resources.
In this paper, we describe the BrainScape framework's architecture and functionalities, 
along with details on the BrainScape dataset, 
aiming to empower the neuroscience community to build inclusive, 
large-scale datasets that facilitate robust computational approaches 
and advance our understanding of brain structure and function.
