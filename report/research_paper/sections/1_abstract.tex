Anatomical MRI data is essential for investigating neurological disorders and understanding brain function. 
However, progress is often hindered by challenges related to data accessibility, small dataset sizes, 
data fragmentation, and inconsistencies across publicly available datasets. Although thousands of high-quality, 
open-source MRI datasets are available for download, their limited size and heterogeneous organization hinder 
effective pooling efforts. Additionally, repackaging and repooling datasets can result in the loss of original 
source information and introduce biases through duplicated images. 
To address these issues, we introduce an open-source Python framework that automates the collection, integration, 
and standardized preprocessing of anatomical MRI scans from \NumDatasets\ diverse datasets, encompassing a total 
of \TotalSubjectsIncludedAfterInspectionCount\ subjects after quality control. The system leverages a plugin-based 
architecture featuring distinct modules for data downloading, file mapping, validation, preprocessing, and demographics 
attachment. The framework supports dataset-specific configurations and, to date, has accommodated \NumDatasets\ 
different datasets, each varying in size, modality, structure, and demographic information, while also facilitating 
the seamless integration of new datasets. Furthermore, each MRI image can be traced back to its original source project 
and repository. Any subjects excluded from datasets are documented in dedicated dataset-specific configuration files, ensuring transparent 
and reproducible exclusion criteria. This unified workflow not only reduces manual labor but also significantly reduces 
the risk of duplication and biases.
We demonstrate the effectiveness of our framework in aggregating multiple MRI modalities (T1w, T2w, and FLAIR) from 
many different sources and integrating essential demographic fields (age, sex, handedness, etc.) for large-scale 
studies. By providing automated, transparent, and configurable workflows, our framework addresses critical challenges 
in open science, accelerates data-driven investigations, and promotes inclusivity and reproducibility in neuroscience research.




