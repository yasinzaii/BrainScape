% Supplementary Material

We provide a detailed table summarizing essential metadata for each dataset included in BrainScape. 
Table \ref{suppDataTable} provides information on the \NumDatasets\ MRI studies and datasets currently included in this framework.
As the project evolves, additional datasets will be added, further expanding this list. 
The following section describes the columns of the table.

\begin{enumerate}
    \item \textbf{Identifier}: Each dataset is assigned a unique identifier (e.g., ``AOMIC'', ``QTAB'') to facilitate referencing within this framework. 
    All MRI scans and associated metadata for each dataset are organized in dedicated folders named after their identifiers, ensuring traceability to the original source or study. 
    Hyperlinks embedded in the identifier column provide direct access to the corresponding source dataset repository.
    Some datasets may require authorization or the completion of specific user agreements for access.

    \item \textbf{Dataset Name}: This column lists the name of the dataset or study from which the anatomical MRI scans were obtained. 
    When applicable, a citation is included to acknowledge the contributions of the original dataset creators.

    \item \textbf{License}: This column specifies the licensing terms under which each dataset is made available. 
    Many datasets obtained through OpenNeuro (\cite{markiewicz2021openneuro}) are distributed under a CC0 license, allowing unrestricted reuse.
    However, some data sources (e.g., ABIDE (\cite{di2014autism}), ABIDE2 (\cite{di2017enhancing}), ADHD200 (\cite{adhd2012adhd}), CORR (\cite{zuo2014open}), CMIHBN (\cite{alexander2017open}), INDINKI (\cite{nooner2012nki})) are governed by CC-BY-NC or similarly restrictive licenses. 
    These datasets often require a signed user agreement, an approved access application, or both. 
    Users are responsible for reviewing and complying with all applicable licensing conditions before using these datasets. 
    
    \item \textbf{Subjects}:  This column reports the number of participants whose anatomical scans are included in this study.
    In some instances, this total may be lower than the number of participants in the original dataset because scans that failed visual quality inspections were excluded. 
    Records of these excluded subjects are maintained in the corresponding dataset metadata, allowing researchers to re-evaluate or reverse exclusion decisions if necessary.

    \item \textbf{T1w, T2w, and FLAIR}: These columns display the number of scans available for each MRI modality in every dataset. 
    Individual subjects may undergo multiple scanning sessions, which can result in scan counts exceeding the total number of subjects. 
\end{enumerate}

Note: gadolinium-enhanced T1-weighted (T1Gd) MRIs are only provided by the BRATS dataset, therefore this modality is not included in Table \ref{suppDataTable}.

% Inputting the Generated Suplimentary Table
\begin{center}
\small
\begin{longtable}{@{}lp{8.5cm}p{1.4cm}llll@{}}
    \caption{Comprehensive List of Integrated Anatomical MRI Datasets} \label{suppDataTable} \\
    \toprule
    \textbf{Identifier} & \textbf{Dataset Name} & \textbf{License} & \textbf{Subjects} & \textbf{T1w} & \textbf{T2w} & \textbf{Flair} \\
    \midrule
    \endfirsthead
    
    \multicolumn{7}{c}{{\bfseries \tablename\ \thetable{} -- continued from previous page}} \\
    \toprule
    \textbf{Identifier} & \textbf{Dataset Name} & \textbf{License} & \textbf{Subjects} & \textbf{T1w} & \textbf{T2w} & \textbf{Flair} \\
    \midrule
    \endhead
    
    \midrule \multicolumn{7}{r}{{Continued on next page}} \\
    \endfoot
    \bottomrule
    \endlastfoot
    
    \mbox{\href{https://www.nitrc.org/ir/data/projects/ABIDE}{\hspace{0.1em}\rule{0pt}{1.2em}ABIDE\rule{0pt}{1.2em}\hspace{0.1em}}} & Autism Brain Imaging Data Exchange I - ABIDE I (\cite{di2014autism}) & CC BY-NC-SA 3.0 & 1032 & 1032 & 0 & 0 \\
    \mbox{\href{https://www.nitrc.org/ir/data/projects/adhd_200}{\hspace{0.1em}\rule{0pt}{1.2em}ADHD200\rule{0pt}{1.2em}\hspace{0.1em}}} & ADHD-200 (\cite{adhd2012adhd}) & CC BY-NC 4.0 & 906 & 906 & 0 & 0 \\
    \mbox{\href{https://openneuro.org/datasets/ds005012/versions/1.0.3}{\hspace{0.1em}\rule{0pt}{1.2em}AHRBS\rule{0pt}{1.2em}\hspace{0.1em}}} & Adolescent Health Risk Behavior Study: Monetary Incentive Delay (MID) task data (\cite{demidenko2024impact}) & CC0 & 60 & 120 & 0 & 0 \\
    \mbox{\href{https://openneuro.org/datasets/ds003097/versions/1.2.1}{\hspace{0.1em}\rule{0pt}{1.2em}AOMIC\rule{0pt}{1.2em}\hspace{0.1em}}} & Amsterdam Open MRI Collection (AOMIC) - AOMIC-ID1000 (\cite{snoek2021amsterdam}) & CC0 & 926 & 2762 & 0 & 0 \\
    \mbox{\href{https://openneuro.org/datasets/ds003436/versions/1.0.0}{\hspace{0.1em}\rule{0pt}{1.2em}APTSIE\rule{0pt}{1.2em}\hspace{0.1em}}} & Agreeableness personality trait and social information encoding (\cite{arbula2021representation}) & CC0 & 55 & 55 & 0 & 0 \\
    \mbox{\href{https://openneuro.org/datasets/ds004884/versions/1.0.2}{\hspace{0.1em}\rule{0pt}{1.2em}ARCD\rule{0pt}{1.2em}\hspace{0.1em}}} & Aphasia Recovery Cohort (ARC) Dataset (\cite{gibson2024aphasia}) & CC0 & 215 & 419 & 412 & 221 \\
    \mbox{\href{https://openneuro.org/datasets/ds002382/versions/1.0.1}{\hspace{0.1em}\rule{0pt}{1.2em}ARDACA\rule{0pt}{1.2em}\hspace{0.1em}}} & Age-related differences in auditory cortex activity during spoken word recognition (\cite{rogers2020age}) & CC0 & 61 & 61 & 61 & 0 \\
    \mbox{\href{https://openneuro.org/datasets/ds005386/versions/1.0.0}{\hspace{0.1em}\rule{0pt}{1.2em}ATTEXP\rule{0pt}{1.2em}\hspace{0.1em}}} & AttExp{\_}fMRI (\cite{penalver2024context}) & CC0 & 52 & 52 & 52 & 0 \\
    \mbox{\href{https://openneuro.org/datasets/ds001796/versions/1.7.0}{\hspace{0.1em}\rule{0pt}{1.2em}BATB\rule{0pt}{1.2em}\hspace{0.1em}}} & Bilingualism and the brain (\cite{deluca2019redefining}) & CC0 & 64 & 64 & 0 & 0 \\
    \mbox{\href{https://openneuro.org/datasets/ds001486/versions/1.3.1}{\hspace{0.1em}\rule{0pt}{1.2em}BCMD\rule{0pt}{1.2em}\hspace{0.1em}}} & Brain Correlates of Math Development (\cite{suarez2019longitudinal}) & CC0 & 126 & 186 & 0 & 0 \\
    \mbox{\href{https://openneuro.org/datasets/ds000208/versions/1.0.1}{\hspace{0.1em}\rule{0pt}{1.2em}BCPPR\rule{0pt}{1.2em}\hspace{0.1em}}} & Brain connectivity predicts placebo response across chronic pain clinical trials (\cite{tetreault2016brain}) & CC0 & 74 & 74 & 0 & 0 \\
    \mbox{\href{https://openneuro.org/datasets/ds004302/versions/1.0.1}{\hspace{0.1em}\rule{0pt}{1.2em}BCSP\rule{0pt}{1.2em}\hspace{0.1em}}} & Brain correlates of speech perception in schizophrenia patients with and without auditory hallucinations (\cite{soler2022brain}) & CC0 & 71 & 71 & 0 & 0 \\
    \mbox{\href{https://openneuro.org/datasets/ds002886/versions/1.1.0}{\hspace{0.1em}\rule{0pt}{1.2em}BDDR\rule{0pt}{1.2em}\hspace{0.1em}}} & Brain Development of Deductive Reasoning (\cite{lytle2020neuroimaging}) & CC0 & 53 & 53 & 0 & 0 \\
    \mbox{\href{https://openneuro.org/datasets/ds003877/versions/1.1.1}{\hspace{0.1em}\rule{0pt}{1.2em}BEANS\rule{0pt}{1.2em}\hspace{0.1em}}} & BrainMorphometry DiminishedGrowth BEAN study 2021 (\cite{turesky2021brain}) & CC0 & 71 & 71 & 0 & 0 \\
    \mbox{\href{https://openneuro.org/datasets/ds004946/versions/1.0.0}{\hspace{0.1em}\rule{0pt}{1.2em}BMDMS\rule{0pt}{1.2em}\hspace{0.1em}}} & Brain mechanisms disciminating enactive mental simulations of runnning and plogging (\cite{philips2024brain}) & CC0 & 90 & 90 & 0 & 0 \\
    \mbox{\href{https://openneuro.org/datasets/ds003798/versions/1.0.5}{\hspace{0.1em}\rule{0pt}{1.2em}CCCS\rule{0pt}{1.2em}\hspace{0.1em}}} & Caltech Conte Center - A multimodal data resource for exploring social cognition and decision-making. (\cite{kliemann2022caltech}) & CC0 & 117 & 249 & 73 & 0 \\
    \mbox{\href{https://openneuro.org/datasets/ds004556/versions/1.0.1}{\hspace{0.1em}\rule{0pt}{1.2em}CCNFT\rule{0pt}{1.2em}\hspace{0.1em}}} & The role of superstition of cognitive control during neurofeedback training - part 1 (\cite{grossinger2021role}) & CC0 & 53 & 53 & 53 & 0 \\
    \mbox{\href{https://openneuro.org/datasets/ds001818/versions/1.0.0}{\hspace{0.1em}\rule{0pt}{1.2em}CFME\rule{0pt}{1.2em}\hspace{0.1em}}} & CAT Faces MRI experiment  & CC0 & 50 & 76 & 0 & 0 \\
    \mbox{\href{https://openneuro.org/datasets/ds004299/versions/1.0.0}{\hspace{0.1em}\rule{0pt}{1.2em}CHLStudy\rule{0pt}{1.2em}\hspace{0.1em}}} & Characterizing habit learning in the human brain at the individual and group levels: a multi-modal MRI study (\cite{gera2023characterizing}) & CC0 & 123 & 223 & 0 & 0 \\
    \mbox{\href{https://openneuro.org/datasets/ds003242/versions/1.0.0}{\hspace{0.1em}\rule{0pt}{1.2em}CICT\rule{0pt}{1.2em}\hspace{0.1em}}} & Cue Induced Craving task Dataset (\cite{tomova2020acute}) & CC0 & 96 & 96 & 96 & 0 \\
    \mbox{\href{https://openneuro.org/datasets/ds004073/versions/1.0.1}{\hspace{0.1em}\rule{0pt}{1.2em}CLLD\rule{0pt}{1.2em}\hspace{0.1em}}} &   & CC0 & 51 & 51 & 0 & 0 \\
    \mbox{\href{https://openneuro.org/datasets/ds003653/versions/1.0.0}{\hspace{0.1em}\rule{0pt}{1.2em}CMDD\rule{0pt}{1.2em}\hspace{0.1em}}} & Cortical myelin measured by the T1w/T2w ratio in individuals with depressive disorders and healthy controls (\cite{baranger2021aberrant}) & CC0 & 87 & 88 & 88 & 0 \\
    \mbox{\href{https://www.nitrc.org/ir/data/projects/corr}{\hspace{0.1em}\rule{0pt}{1.2em}CORR\rule{0pt}{1.2em}\hspace{0.1em}}} & Consortium for Reliability and Reproducibility (CoRR) (\cite{zuo2014open}) & CC BY 4.0 & 1346 & 1478 & 0 & 0 \\
    \mbox{\href{https://openneuro.org/datasets/ds003701/versions/1.0.1}{\hspace{0.1em}\rule{0pt}{1.2em}CPRO\rule{0pt}{1.2em}\hspace{0.1em}}} & Concrete Permuted Rule Operations (\cite{ito2017cognitive}) & CC0 & 94 & 94 & 94 & 0 \\
    \mbox{\href{https://openneuro.org/datasets/ds004604/versions/2.0.0}{\hspace{0.1em}\rule{0pt}{1.2em}CRND\rule{0pt}{1.2em}\hspace{0.1em}}} & Cerebrovascular Reactivity Normative Dataset (\cite{rovai2024cvrmap}) & CC0 & 50 & 50 & 0 & 0 \\
    \mbox{\href{https://openneuro.org/datasets/ds002236/versions/1.1.1}{\hspace{0.1em}\rule{0pt}{1.2em}CSMLP\rule{0pt}{1.2em}\hspace{0.1em}}} & Cross-Sectional Multidomain Lexical Processing (\cite{lytle2020neuroimaging}) & CC0 & 73 & 73 & 0 & 0 \\
    \mbox{\href{https://openneuro.org/datasets/ds004261/versions/2.0.0}{\hspace{0.1em}\rule{0pt}{1.2em}CSNPS\rule{0pt}{1.2em}\hspace{0.1em}}} & Cross-stage neural pattern similarity in the hippocampus predicts false memory derived from post-event inaccurate information (\cite{shao2023cross}) & CC0 & 55 & 55 & 0 & 0 \\
    \mbox{\href{https://openneuro.org/datasets/ds003831/versions/1.0.0}{\hspace{0.1em}\rule{0pt}{1.2em}CTM\rule{0pt}{1.2em}\hspace{0.1em}}} & Cognitive Control Theoretic Mechanisms of Real-time fMRI-Guided Neuromodulation (CTM) (\cite{bush2022action}) & CC0 & 73 & 73 & 0 & 0 \\
    \mbox{\href{https://openneuro.org/datasets/ds002843/versions/1.0.1}{\hspace{0.1em}\rule{0pt}{1.2em}CTS\rule{0pt}{1.2em}\hspace{0.1em}}} & Cognitive Training (\cite{kable2017no}) & CC0 & 157 & 278 & 275 & 0 \\
    \mbox{\href{https://openneuro.org/datasets/ds004636/versions/1.0.4}{\hspace{0.1em}\rule{0pt}{1.2em}CTStudy\rule{0pt}{1.2em}\hspace{0.1em}}} & Cognitive tasks, anatomical MRI, and functional MRI data evaluating the construct of self-regulation (\cite{bissett2024cognitive}) & CC0 & 107 & 179 & 138 & 0 \\
    \mbox{\href{https://openneuro.org/datasets/ds002643/versions/1.1.0}{\hspace{0.1em}\rule{0pt}{1.2em}CWHSH\rule{0pt}{1.2em}\hspace{0.1em}}} & Can we have a second helping? A replication study on the neurobiological mechanisms underlying self-control (\cite{scholz2022can}) & CC0 & 80 & 80 & 0 & 0 \\
    \mbox{\href{https://openneuro.org/datasets/ds003499/versions/1.0.1}{\hspace{0.1em}\rule{0pt}{1.2em}DCPC\rule{0pt}{1.2em}\hspace{0.1em}}} & Developmental change in prefrontal cortex recruitment supports the emergence of value-guided memory (\cite{nussenbaum2021developmental}) & CC0 & 90 & 93 & 88 & 0 \\
    \mbox{\href{https://openneuro.org/datasets/ds001748/versions/1.0.4}{\hspace{0.1em}\rule{0pt}{1.2em}DFN\rule{0pt}{1.2em}\hspace{0.1em}}} & Differentiation of functional networks during long-term memory retrieval in children and adolescents (\cite{fynes2019differentiation}) & CC0 & 60 & 60 & 0 & 0 \\
    \mbox{\href{https://openneuro.org/datasets/ds004856/versions/1.2.0}{\hspace{0.1em}\rule{0pt}{1.2em}DLBStudy\rule{0pt}{1.2em}\hspace{0.1em}}} & The Dallas Lifespan Brain Study (\cite{mcdonough2016discrepancies}) & CC0 & 393 & 800 & 0 & 805 \\
    \mbox{\href{https://openneuro.org/datasets/ds005518/versions/1.0.1}{\hspace{0.1em}\rule{0pt}{1.2em}DNRNP\rule{0pt}{1.2em}\hspace{0.1em}}} & Deciphering the neural responses to a naturalistic persuasive message (\cite{ntoumanis2024deciphering}) & CC0 & 50 & 50 & 0 & 0 \\
    \mbox{\href{https://openneuro.org/datasets/ds005529/versions/1.0.1}{\hspace{0.1em}\rule{0pt}{1.2em}DPCASL\rule{0pt}{1.2em}\hspace{0.1em}}} & DP-pCASL data (CBF, ATT, BBB kw) from 186 cognitively normal participants (8-92 years) (\cite{shao2024age}) & CC0 & 49 & 49 & 0 & 0 \\
    \mbox{\href{https://openneuro.org/datasets/ds002320/versions/1.1.0}{\hspace{0.1em}\rule{0pt}{1.2em}DPTStudy\rule{0pt}{1.2em}\hspace{0.1em}}} & Dynamic Passive Threat (\cite{meyer2019dynamic}) & CC0 & 72 & 72 & 72 & 0 \\
    \mbox{\href{https://openneuro.org/datasets/ds002116/versions/1.0.0}{\hspace{0.1em}\rule{0pt}{1.2em}DSNP\rule{0pt}{1.2em}\hspace{0.1em}}} & Development of Symbolic Number Processing  & CC0 & 55 & 55 & 0 & 0 \\
    \mbox{\href{https://openneuro.org/datasets/ds004513/versions/1.0.4}{\hspace{0.1em}\rule{0pt}{1.2em}ECStudy\rule{0pt}{1.2em}\hspace{0.1em}}} & The Energetic Costs of the Human Connectome (\cite{castrillon2023energy}) & CC0 & 19 & 28 & 0 & 0 \\
    \mbox{\href{https://openneuro.org/datasets/ds004605/versions/1.0.1}{\hspace{0.1em}\rule{0pt}{1.2em}EDBP\rule{0pt}{1.2em}\hspace{0.1em}}} & Emotion and Development Branch Phenotyping and DTI (2012-2017) (\cite{mckay2024modeling}) & CC0 & 129 & 129 & 130 & 0 \\
    \mbox{\href{https://openneuro.org/datasets/ds001242/versions/1.0.0}{\hspace{0.1em}\rule{0pt}{1.2em}EEAR\rule{0pt}{1.2em}\hspace{0.1em}}} & Examining effects of arousal on responses to salient and non-salient stimuli in younger and older adults (\cite{lee2018arousal}) & CC0 & 49 & 49 & 0 & 0 \\
    \mbox{\href{https://openneuro.org/datasets/ds004349/versions/1.0.0}{\hspace{0.1em}\rule{0pt}{1.2em}EMMBC\rule{0pt}{1.2em}\hspace{0.1em}}} & Evaluating methods for measuring background connectivity in slow event-related fMRI designs (\cite{frank2023evaluating}) & CC0 & 56 & 56 & 0 & 0 \\
    \mbox{\href{https://openneuro.org/datasets/ds002620/versions/1.0.0}{\hspace{0.1em}\rule{0pt}{1.2em}ERAB\rule{0pt}{1.2em}\hspace{0.1em}}} & Emotion regulation in the Ageing Brain, University of Reading, BBSRC (\cite{lloyd2021longitudinal}) & CC0 & 78 & 78 & 0 & 0 \\
    \mbox{\href{https://openneuro.org/datasets/ds004144/versions/1.0.2}{\hspace{0.1em}\rule{0pt}{1.2em}ERFF\rule{0pt}{1.2em}\hspace{0.1em}}} & A behavioral, clinical and brain imaging dataset with focus on emotion regulation of females with fibromyalgia (\cite{balducci2022behavioral}) & CC0 & 66 & 66 & 66 & 0 \\
    \mbox{\href{https://openneuro.org/datasets/ds004697/versions/1.0.2}{\hspace{0.1em}\rule{0pt}{1.2em}FBS\rule{0pt}{1.2em}\hspace{0.1em}}} & Food and Brain Study (\cite{keller2023children}) & CC0 & 78 & 78 & 0 & 0 \\
    \mbox{\href{https://www.nitrc.org/projects/fcon_1000/}{\hspace{0.1em}\rule{0pt}{1.2em}FC1000\rule{0pt}{1.2em}\hspace{0.1em}}} & The 1000 Functional Connectomes Project (\cite{biswal2010toward}) & CC BY-NC 4.0 & 861 & 861 & 0 & 0 \\
    \mbox{\href{https://openneuro.org/datasets/ds005215/versions/1.0.0}{\hspace{0.1em}\rule{0pt}{1.2em}FDNPS\rule{0pt}{1.2em}\hspace{0.1em}}} & NarrativePuzzle  & CC0 & 65 & 65 & 0 & 0 \\
    \mbox{\href{https://openneuro.org/datasets/ds004848/versions/1.0.1}{\hspace{0.1em}\rule{0pt}{1.2em}GOTD\rule{0pt}{1.2em}\hspace{0.1em}}} & Game of Thrones - A naturalistic viewing dataset (\cite{noad2024familiarity}) & CC0 & 73 & 73 & 0 & 0 \\
    \mbox{\href{https://openneuro.org/datasets/ds004791/versions/1.0.0}{\hspace{0.1em}\rule{0pt}{1.2em}HBM\rule{0pt}{1.2em}\hspace{0.1em}}} & Kwok et al., 2023, Human Brain Mapping (\cite{kwok2023developmental}) & CC0 & 66 & 66 & 0 & 0 \\
    \mbox{\href{https://www.humanconnectome.org/study/hcp-young-adult}{\hspace{0.1em}\rule{0pt}{1.2em}HCP1200\rule{0pt}{1.2em}\hspace{0.1em}}} & Human Connectome Project 1200 Subjects Data Release (\cite{van2013wu}) & - & 1113 & 1113 & 1113 & 0 \\
    \mbox{\href{https://openneuro.org/datasets/ds005026/versions/1.0.0}{\hspace{0.1em}\rule{0pt}{1.2em}HLC\rule{0pt}{1.2em}\hspace{0.1em}}} & Hearing loss Connectome (\cite{ponticorvo2022cross}) & CC0 & 82 & 82 & 0 & 0 \\
    \mbox{\href{https://openneuro.org/datasets/ds003823/versions/1.3.3}{\hspace{0.1em}\rule{0pt}{1.2em}HRVStudy\rule{0pt}{1.2em}\hspace{0.1em}}} & Heart rate variability biofeedback training and emotion regulation (\cite{min2022emotion}) & CC0 & 171 & 315 & 0 & 0 \\
    \mbox{\href{https://openneuro.org/datasets/ds001838/versions/1.0.1}{\hspace{0.1em}\rule{0pt}{1.2em}HSNP\rule{0pt}{1.2em}\hspace{0.1em}}} & Handedness and Symbolic Number Representation (\cite{goffin2019does}) & CC0 & 53 & 53 & 0 & 0 \\
    \mbox{\href{https://openneuro.org/datasets/ds002513/versions/1.0.0}{\hspace{0.1em}\rule{0pt}{1.2em}IBRGU\rule{0pt}{1.2em}\hspace{0.1em}}} & Increased brain reactivity to gambling unavailability is a marker of problem gambling (\cite{brevers2021increased}) & CC0 & 65 & 65 & 0 & 0 \\
    \mbox{\href{https://openneuro.org/datasets/ds003763/versions/1.0.5}{\hspace{0.1em}\rule{0pt}{1.2em}IDAS\rule{0pt}{1.2em}\hspace{0.1em}}} & Interoception during aging: The heartbeat detection task (\cite{dobrushina2020ability}) & CC0 & 39 & 39 & 0 & 0 \\
    \mbox{\href{https://openneuro.org/datasets/ds005602/versions/1.0.0}{\hspace{0.1em}\rule{0pt}{1.2em}IDEAS\rule{0pt}{1.2em}\hspace{0.1em}}} & The Imaging Database for Epilepsy And Surgery (IDEAS) (\cite{taylor2024imaging}) & CC0 & 365 & 365 & 0 & 317 \\
    \mbox{\href{https://openneuro.org/datasets/ds003849/versions/1.0.0}{\hspace{0.1em}\rule{0pt}{1.2em}IDFPH\rule{0pt}{1.2em}\hspace{0.1em}}} & Individual differences in frontoparietal plasticity in humans (\cite{boroshok2022individual}) & CC0 & 92 & 92 & 92 & 0 \\
    \mbox{\href{https://openneuro.org/datasets/ds004589/versions/1.0.0}{\hspace{0.1em}\rule{0pt}{1.2em}IDMA\rule{0pt}{1.2em}\hspace{0.1em}}} & fMRI Investigations of Individual Differences on Memory Activation for Faces and Words: The Effects of Handedness and Phenomenal Experience (\cite{schmidt2024memory}) & CC0 & 98 & 98 & 0 & 0 \\
    \mbox{\href{https://openneuro.org/datasets/ds002647/versions/1.0.1}{\hspace{0.1em}\rule{0pt}{1.2em}IEFA\rule{0pt}{1.2em}\hspace{0.1em}}} & Isometric exercise facilitates attention to salient events in women via the noradrenergic system (\cite{mather2020isometric}) & CC0 & 91 & 91 & 0 & 0 \\
    \mbox{\href{https://openneuro.org/datasets/ds003612/versions/1.0.4}{\hspace{0.1em}\rule{0pt}{1.2em}IHNC\rule{0pt}{1.2em}\hspace{0.1em}}} & The impact of handedness on the neural correlates during kinesthetic motor imagery: a FMRI study (\cite{crotti2022handedness}) & CC0 & 51 & 51 & 0 & 0 \\
    \mbox{\href{https://openneuro.org/datasets/ds005455/versions/1.1.5}{\hspace{0.1em}\rule{0pt}{1.2em}ILCCB\rule{0pt}{1.2em}\hspace{0.1em}}} & An fMRI dataset for investigating language control and cognitive control in bilinguals  & CC0 & 75 & 75 & 0 & 0 \\
    \mbox{\href{https://www.nitrc.org/ir/data/projects/nki_rockland}{\hspace{0.1em}\rule{0pt}{1.2em}INDINKI\rule{0pt}{1.2em}\hspace{0.1em}}} & INDI NKI/Rockland Sample (\cite{nooner2012nki}) & Attribution Non-Commercial & 201 & 202 & 0 & 0 \\
    \mbox{\href{https://openneuro.org/datasets/ds004259/versions/1.0.0}{\hspace{0.1em}\rule{0pt}{1.2em}IRAS\rule{0pt}{1.2em}\hspace{0.1em}}} & Individual risk attitudes arise from noise in neurocognitive magnitude representations. (\cite{garcia2022individual}) & CC0 & 64 & 64 & 0 & 0 \\
    \mbox{\href{https://openneuro.org/datasets/ds004217/versions/1.0.0}{\hspace{0.1em}\rule{0pt}{1.2em}IRMS\rule{0pt}{1.2em}\hspace{0.1em}}} & Interacting representations of mental states and traits  & CC0 & 53 & 53 & 0 & 0 \\
    \mbox{\href{https://www.nitrc.org/ir/data/projects/ixi}{\hspace{0.1em}\rule{0pt}{1.2em}IXI\rule{0pt}{1.2em}\hspace{0.1em}}} & IXI (Information eXtraction from Images) dataset  &  CC BY-SA 3.0 & 401 & 400 & 397 & 0 \\
    \mbox{\href{https://openneuro.org/datasets/ds001894/versions/1.4.2}{\hspace{0.1em}\rule{0pt}{1.2em}LBCMLP\rule{0pt}{1.2em}\hspace{0.1em}}} & Longitudinal Brain Correlates of Multisensory Lexical Processing in Children (\cite{lytle2019longitudinal}) & CC0 & 184 & 233 & 0 & 0 \\
    \mbox{\href{https://openneuro.org/datasets/ds003508/versions/1.0.0}{\hspace{0.1em}\rule{0pt}{1.2em}LLAD\rule{0pt}{1.2em}\hspace{0.1em}}} & Language Learning Aptitude dataset (\cite{noven2021cortical}) & CC0 & 57 & 57 & 0 & 0 \\
    \mbox{\href{https://openneuro.org/datasets/ds004718/versions/1.1.0}{\hspace{0.1em}\rule{0pt}{1.2em}LPPHK\rule{0pt}{1.2em}\hspace{0.1em}}} & Le Petit Prince Hong Kong - Naturalistic fMRI and EEG dataset from older Cantonese speakers (\cite{momenian2024petit}) & CC0 & 50 & 50 & 0 & 0 \\
    \mbox{\href{https://openneuro.org/datasets/ds003643/versions/2.0.5}{\hspace{0.1em}\rule{0pt}{1.2em}LPPStudy\rule{0pt}{1.2em}\hspace{0.1em}}} & Le Petit Prince - A multilingual fMRI corpus using ecological stimuli (\cite{li2022petit}) & CC0 & 93 & 93 & 0 & 0 \\
    \mbox{\href{https://openneuro.org/datasets/ds004553/versions/1.0.1}{\hspace{0.1em}\rule{0pt}{1.2em}LRES\rule{0pt}{1.2em}\hspace{0.1em}}} & Learning rules of engagement for social exchange within and between groups (\cite{rojek2023learning}) & CC0 & 50 & 50 & 0 & 0 \\
    \mbox{\href{https://openneuro.org/datasets/ds005366/versions/1.2.0}{\hspace{0.1em}\rule{0pt}{1.2em}LSFD\rule{0pt}{1.2em}\hspace{0.1em}}} & Large-scale fMRI dataset for the design of motor-based Brain-Computer Interfaces (\cite{bom2024large}) & CC0 & 137 & 137 & 0 & 0 \\
    \mbox{\href{https://openneuro.org/datasets/ds003949/versions/1.0.1}{\hspace{0.1em}\rule{0pt}{1.2em}LTMAC\rule{0pt}{1.2em}\hspace{0.1em}}} & Lausanne{\_}TOF-MRA{\_}Aneurysm{\_}Cohort (\cite{di2023towards}) & CC0 & 278 & 290 & 0 & 0 \\
    \mbox{\href{https://openneuro.org/datasets/ds004285/versions/1.0.0}{\hspace{0.1em}\rule{0pt}{1.2em}LTS\rule{0pt}{1.2em}\hspace{0.1em}}} & Listening task (\cite{rogers2023real}) & CC0 & 78 & 78 & 78 & 0 \\
    \mbox{\href{https://openneuro.org/datasets/ds004935}{\hspace{0.1em}\rule{0pt}{1.2em}MAICY\rule{0pt}{1.2em}\hspace{0.1em}}} & Multivariate Assessment of Inhibitory Control in Youth: Links with Psychopathology and Brain Function Dataset (\cite{cardinale2024multivariate}) & CC0 & 118 & 118 & 0 & 0 \\
    \mbox{\href{https://openneuro.org/datasets/ds005038/versions/1.0.3}{\hspace{0.1em}\rule{0pt}{1.2em}MBMP\rule{0pt}{1.2em}\hspace{0.1em}}} & Modality-based Multitasking and Practice - fMRI (\cite{mueckstein2024multitasking}) & CC0 & 57 & 57 & 0 & 0 \\
    \mbox{\href{https://openneuro.org/datasets/ds005016/versions/1.1.1}{\hspace{0.1em}\rule{0pt}{1.2em}MBSR\rule{0pt}{1.2em}\hspace{0.1em}}} & Modality-based Multitasking and Practice - fMRI (\cite{seminowicz2020enhanced}) & CC0 & 146 & 344 & 0 & 0 \\
    \mbox{\href{https://openneuro.org/datasets/ds000119/versions/00001}{\hspace{0.1em}\rule{0pt}{1.2em}MCAC\rule{0pt}{1.2em}\hspace{0.1em}}} & Maturational Changes in Anterior Cingulate and Frontoparietal Recruitment Support the Development of Error Processing and Inhibitory Control (Antistate) (\cite{velanova2008maturational}) & CC0 & 70 & 70 & 0 & 0 \\
    \mbox{\href{https://openneuro.org/datasets/ds004466/versions/1.0.2}{\hspace{0.1em}\rule{0pt}{1.2em}MCS\rule{0pt}{1.2em}\hspace{0.1em}}} & Mapping the Connectome of Synaesthesia - An Open Access MRI Dataset (\cite{racey2023open}) & CC0 & 127 & 127 & 127 & 0 \\
    \mbox{\href{https://openneuro.org/datasets/ds004455/versions/1.1.0}{\hspace{0.1em}\rule{0pt}{1.2em}MCStudy\rule{0pt}{1.2em}\hspace{0.1em}}} & magic{\_}carpet (\cite{feher2023rethinking}) & CC0 & 94 & 94 & 0 & 0 \\
    \mbox{\href{https://openneuro.org/datasets/ds000258/versions/1.0.1}{\hspace{0.1em}\rule{0pt}{1.2em}MEC\rule{0pt}{1.2em}\hspace{0.1em}}} & Multi-echo Cambridge (\cite{power2018ridding}) & CC0 & 89 & 89 & 0 & 0 \\
    \mbox{\href{https://openneuro.org/datasets/ds004499/versions/1.0.3}{\hspace{0.1em}\rule{0pt}{1.2em}MESD\rule{0pt}{1.2em}\hspace{0.1em}}} & Multi-echo simultaneous multislice fMRI dataset: Effect of acquisition parameters on fMRI data  & CC0 & 50 & 50 & 0 & 0 \\
    \mbox{\href{https://openneuro.org/datasets/ds002731/versions/1.0.2}{\hspace{0.1em}\rule{0pt}{1.2em}MIMR\rule{0pt}{1.2em}\hspace{0.1em}}} & Multiple interactive memory representations underlie the induction of false memory  & CC0 & 59 & 59 & 0 & 0 \\
    \mbox{\href{https://openneuro.org/datasets/ds005027/versions/1.0.3}{\hspace{0.1em}\rule{0pt}{1.2em}MLStudy\rule{0pt}{1.2em}\hspace{0.1em}}} & Michigan Longitudinal Study (\cite{yau2012nucleus}) & CC0 & 72 & 150 & 0 & 0 \\
    \mbox{\href{https://openneuro.org/datasets/ds004182/versions/1.0.1}{\hspace{0.1em}\rule{0pt}{1.2em}MMC\rule{0pt}{1.2em}\hspace{0.1em}}} & Magic, Memory, and Curiosity (MMC) fMRI Dataset (\cite{ozono2021magic}) & CC0 & 50 & 50 & 0 & 0 \\
    \mbox{\href{https://openneuro.org/datasets/ds000221/versions/00002}{\hspace{0.1em}\rule{0pt}{1.2em}MPLMBB\rule{0pt}{1.2em}\hspace{0.1em}}} & The MPI-Leipzig Mind-Brain-Body dataset (\cite{babayan2019mind}) & CC0 & 318 & 319 & 226 & 306 \\
    \mbox{\href{https://openneuro.org/datasets/ds004173/versions/1.0.2}{\hspace{0.1em}\rule{0pt}{1.2em}MRART\rule{0pt}{1.2em}\hspace{0.1em}}} & Movement-related artefacts (MR-ART) dataset (\cite{narai2022movement}) & CC0 & 148 & 148 & 0 & 0 \\
    \mbox{\href{https://openneuro.org/datasets/ds001131/versions/1.0.0}{\hspace{0.1em}\rule{0pt}{1.2em}MVD\rule{0pt}{1.2em}\hspace{0.1em}}} & Milky-Vodka  & CC0 & 54 & 54 & 0 & 0 \\
    \mbox{\href{https://openneuro.org/datasets/ds001734/versions/1.0.5}{\hspace{0.1em}\rule{0pt}{1.2em}NARPS\rule{0pt}{1.2em}\hspace{0.1em}}} & NARPS (\cite{botvinik2019fmri}) & CC0 & 108 & 108 & 0 & 0 \\
    \mbox{\href{https://openneuro.org/datasets/ds002345/versions/1.1.4}{\hspace{0.1em}\rule{0pt}{1.2em}NARR\rule{0pt}{1.2em}\hspace{0.1em}}} & Narratives  & CC0 & 334 & 365 & 0 & 0 \\
    \mbox{\href{https://openneuro.org/datasets/ds003592/versions/1.0.13}{\hspace{0.1em}\rule{0pt}{1.2em}NCAS\rule{0pt}{1.2em}\hspace{0.1em}}} & Neurocognitive aging data release with behavioral, structural, and multi-echo functional MRI measures (\cite{setton2023age}) & CC0 & 257 & 257 & 0 & 207 \\
    \mbox{\href{https://openneuro.org/datasets/ds005364/versions/1.0.0}{\hspace{0.1em}\rule{0pt}{1.2em}NCStudy\rule{0pt}{1.2em}\hspace{0.1em}}} & neuroCOVID MRI dWI and fMRI with reversal learning  & CC0 & 100 & 100 & 100 & 0 \\
    \mbox{\href{https://openneuro.org/datasets/ds003764/versions/1.0.5}{\hspace{0.1em}\rule{0pt}{1.2em}NDStudy\rule{0pt}{1.2em}\hspace{0.1em}}} & Functional magnetic resonance imaging data for the neural dynamics underlying the acquisition of distinct auditory categories (\cite{feng2021neural}) & CC0 & 55 & 55 & 0 & 0 \\
    \mbox{\href{https://openneuro.org/datasets/ds003148/versions/1.0.1}{\hspace{0.1em}\rule{0pt}{1.2em}NENST\rule{0pt}{1.2em}\hspace{0.1em}}} & Neuroimaging evidence for network sampling theory of human intelligence (\cite{soreq2021neuroimaging}) & CC0 & 60 & 60 & 0 & 0 \\
    \mbox{\href{https://openneuro.org/datasets/ds004996/versions/1.0.2}{\hspace{0.1em}\rule{0pt}{1.2em}NES\rule{0pt}{1.2em}\hspace{0.1em}}} & NeuroEngage  & CC0 & 51 & 51 & 0 & 0 \\
    \mbox{\href{https://openneuro.org/datasets/ds003709/versions/1.0.0}{\hspace{0.1em}\rule{0pt}{1.2em}NIMHC\rule{0pt}{1.2em}\hspace{0.1em}}} & NIMH-CompyPsych MMI (\cite{keren2021temporal}) & CC0 & 51 & 51 & 0 & 0 \\
    \mbox{\href{https://openneuro.org/datasets/ds004215}{\hspace{0.1em}\rule{0pt}{1.2em}NIMHIHV\rule{0pt}{1.2em}\hspace{0.1em}}} & The NIMH intramural healthy volunteer dataset (\cite{nugent2022nimh}) & CC0 & 247 & 249 & 247 & 241 \\
    \mbox{\href{https://openneuro.org/datasets/ds002837/versions/2.0.0}{\hspace{0.1em}\rule{0pt}{1.2em}NND\rule{0pt}{1.2em}\hspace{0.1em}}} & Naturalistic Neuroimaging Database (\cite{aliko2020naturalistic}) & CC0 & 86 & 86 & 0 & 0 \\
    \mbox{\href{https://openneuro.org/datasets/ds002330/versions/1.1.0}{\hspace{0.1em}\rule{0pt}{1.2em}NPCHA\rule{0pt}{1.2em}\hspace{0.1em}}} & Neuroimaging predictors of creativity in healthy adults (\cite{sunavsky2020neuroimaging}) & CC0 & 66 & 66 & 66 & 0 \\
    \mbox{\href{https://openneuro.org/datasets/ds004327/versions/1.0.3}{\hspace{0.1em}\rule{0pt}{1.2em}OIAStudy\rule{0pt}{1.2em}\hspace{0.1em}}} & Odour-imagery ability is linked to food craving, intake, and adiposity change in humans (\cite{perszyk2023odour}) & CC0 & 46 & 46 & 0 & 0 \\
    \mbox{\href{https://openneuro.org/datasets/ds005504/versions/1.0.0}{\hspace{0.1em}\rule{0pt}{1.2em}PAIC\rule{0pt}{1.2em}\hspace{0.1em}}} & Psychosocial adversity and inhibitory control: an fMRI study of children growing up in extreme poverty (\cite{surani2024examining}) & CC0 & 64 & 64 & 0 & 0 \\
    \mbox{\href{https://openneuro.org/datasets/ds001848/versions/1.0.1}{\hspace{0.1em}\rule{0pt}{1.2em}PASQP\rule{0pt}{1.2em}\hspace{0.1em}}} & Parallel Adaptation of Symbols, Quantities, and Physical Size (\cite{sokolowski2021symbols}) & CC0 & 52 & 52 & 0 & 0 \\
    \mbox{\href{https://openneuro.org/datasets/ds004392/versions/1.0.0}{\hspace{0.1em}\rule{0pt}{1.2em}PDFCC\rule{0pt}{1.2em}\hspace{0.1em}}} & Parkinson's disease, functional connectivity, and cognition (\cite{wylie2023hippocampal}) & CC0 & 57 & 57 & 0 & 0 \\
    \mbox{\href{https://openneuro.org/datasets/ds004917/versions/1.0.1}{\hspace{0.1em}\rule{0pt}{1.2em}PDMTA\rule{0pt}{1.2em}\hspace{0.1em}}} & Probability Decision-making Task with ambiguity (\cite{valdebenito2024parietal}) & CC0 & 52 & 52 & 52 & 0 \\
    \mbox{\href{https://openneuro.org/datasets/ds003721/versions/1.0.1}{\hspace{0.1em}\rule{0pt}{1.2em}PENAStudy\rule{0pt}{1.2em}\hspace{0.1em}}} & A naturalistic paradigm to investigate post-encoding neural activation patterns in relation to subsequent voluntary and intrusive recall of distressing events (\cite{visser2022naturalistic}) & CC0 & 35 & 35 & 0 & 0 \\
    \mbox{\href{https://openneuro.org/datasets/ds004837/versions/1.0.0}{\hspace{0.1em}\rule{0pt}{1.2em}PEPP\rule{0pt}{1.2em}\hspace{0.1em}}} & Magnetoencephalographic (MEG) Pitch and Duration Mismatch Negativity (MMN) in First-Episode Psychosis (\cite{lopez2024source}) & CC0 & 60 & 60 & 0 & 0 \\
    \mbox{\href{https://openneuro.org/datasets/ds003345/versions/1.0.2}{\hspace{0.1em}\rule{0pt}{1.2em}PKIK\rule{0pt}{1.2em}\hspace{0.1em}}} & PenaltyKik.02 (\cite{mcdonald2019bayesian}) & CC0 & 69 & 69 & 0 & 0 \\
    \mbox{\href{https://openneuro.org/datasets/ds003481/versions/1.0.3}{\hspace{0.1em}\rule{0pt}{1.2em}PLS\rule{0pt}{1.2em}\hspace{0.1em}}} & Pragmatic Language (\cite{reyes2023contribution}) & CC0 & 144 & 144 & 0 & 0 \\
    \mbox{\href{https://openneuro.org/datasets/ds005040/versions/1.2.0}{\hspace{0.1em}\rule{0pt}{1.2em}PMM\rule{0pt}{1.2em}\hspace{0.1em}}} & Political Moralization (PMM) (\cite{cohen2024intersubject}) & CC0 & 50 & 50 & 0 & 0 \\
    \mbox{\href{https://openneuro.org/datasets/ds005375/versions/1.0.0}{\hspace{0.1em}\rule{0pt}{1.2em}POLEX\rule{0pt}{1.2em}\hspace{0.1em}}} & POLEX  & CC0 & 58 & 58 & 0 & 0 \\
    \mbox{\href{https://openneuro.org/datasets/ds004746/versions/1.0.1}{\hspace{0.1em}\rule{0pt}{1.2em}PPS\rule{0pt}{1.2em}\hspace{0.1em}}} & Paingen{\_}placebo (\cite{botvinik2024placebo}) & CC0 & 394 & 396 & 0 & 0 \\
    \mbox{\href{https://openneuro.org/datasets/ds004199/versions/1.0.5}{\hspace{0.1em}\rule{0pt}{1.2em}PSED\rule{0pt}{1.2em}\hspace{0.1em}}} & An open presurgery MRI dataset of people with epilepsy and focal cortical dysplasia type II (\cite{schuch2023open}) & CC0 & 168 & 168 & 0 & 168 \\
    \mbox{\href{https://openneuro.org/datasets/ds000202/versions/00001}{\hspace{0.1em}\rule{0pt}{1.2em}PTStudy\rule{0pt}{1.2em}\hspace{0.1em}}} & The heterogeneity in retrieved relations between the personality trait 'Harm avoidance' and gray matter volumes due to variations in the VBM and ROI labeling processing settings (\cite{van2016heterogeneity}) & CC0 & 95 & 95 & 0 & 0 \\
    \mbox{\href{https://openneuro.org/datasets/ds004146/versions/1.0.4}{\hspace{0.1em}\rule{0pt}{1.2em}QTAB\rule{0pt}{1.2em}\hspace{0.1em}}} & Queensland Twin Adolescent Brain (QTAB) (\cite{strike2023queensland}) & CC0 & 274 & 480 & 441 & 449 \\
    \mbox{\href{https://openneuro.org/datasets/ds004169/versions/1.0.7}{\hspace{0.1em}\rule{0pt}{1.2em}QTIM\rule{0pt}{1.2em}\hspace{0.1em}}} & Queensland Twin IMaging (QTIM) (\cite{strike2023queensland}) & CC0 & 1167 & 1301 & 0 & 0 \\
    \mbox{\href{https://openneuro.org/datasets/ds003974/versions/3.0.0}{\hspace{0.1em}\rule{0pt}{1.2em}RBPL1\rule{0pt}{1.2em}\hspace{0.1em}}} & The Reading Brain Project L1 Adults (\cite{li2019reading}) & CC0 & 52 & 52 & 0 & 0 \\
    \mbox{\href{https://openneuro.org/datasets/ds003988/versions/1.0.0}{\hspace{0.1em}\rule{0pt}{1.2em}RBPL2\rule{0pt}{1.2em}\hspace{0.1em}}} & The Reading Brain Project L2 Adults (\cite{li2019reading}) & CC0 & 56 & 56 & 0 & 0 \\
    \mbox{\href{https://openneuro.org/datasets/ds004765/versions/1.0.0}{\hspace{0.1em}\rule{0pt}{1.2em}RRBD\rule{0pt}{1.2em}\hspace{0.1em}}} & Relationship between resting state functional connectivity and reading-related behavioural measures in 69 adults (\cite{bathelt2024relationship}) & CC0 & 71 & 136 & 0 & 0 \\
    \mbox{\href{https://openneuro.org/datasets/ds003126/versions/1.3.1}{\hspace{0.1em}\rule{0pt}{1.2em}RRFAC\rule{0pt}{1.2em}\hspace{0.1em}}} & Reading-related functional activity in children with isolated spelling deficits and dyslexia (\cite{banfi2021reading}) & CC0 & 58 & 58 & 0 & 0 \\
    \mbox{\href{https://openneuro.org/datasets/ds002748/versions/1.0.5}{\hspace{0.1em}\rule{0pt}{1.2em}RSCEP\rule{0pt}{1.2em}\hspace{0.1em}}} & Resting state with closed eyes for patients with depression and healthy participants (\cite{bezmaternykh2021brain}) & CC0 & 70 & 70 & 0 & 0 \\
    \mbox{\href{https://openneuro.org/datasets/ds003871/versions/1.0.2}{\hspace{0.1em}\rule{0pt}{1.2em}RSD\rule{0pt}{1.2em}\hspace{0.1em}}} & Resting-state for 34 younger and 28 older adults (\cite{wahlheim2021connectome}) & CC0 & 58 & 58 & 0 & 0 \\
    \mbox{\href{https://openneuro.org/datasets/ds001747/versions/1.1.0}{\hspace{0.1em}\rule{0pt}{1.2em}RSNA\rule{0pt}{1.2em}\hspace{0.1em}}} & Exploring the Resting State Neural Activity of Monolinguals and Late and Early Bilinguals (\cite{gold2018exploring}) & CC0 & 92 & 92 & 0 & 0 \\
    \mbox{\href{https://openneuro.org/datasets/ds000240/versions/2.0.0}{\hspace{0.1em}\rule{0pt}{1.2em}RSPHA\rule{0pt}{1.2em}\hspace{0.1em}}} & Resting State Perfusion in Healthy Aging (\cite{vidorreta2013comparison}) & CC0 & 60 & 60 & 0 & 0 \\
    \mbox{\href{https://openneuro.org/datasets/ds001408/versions/1.0.3}{\hspace{0.1em}\rule{0pt}{1.2em}RSSE\rule{0pt}{1.2em}\hspace{0.1em}}} & rsfMRI{\_}single{\_}session{\_}EEG{\_}NF (\cite{dobrushina2020modulation}) & CC0 & 51 & 102 & 0 & 0 \\
    \mbox{\href{https://openneuro.org/datasets/ds004725/versions/1.0.1}{\hspace{0.1em}\rule{0pt}{1.2em}SDIOA\rule{0pt}{1.2em}\hspace{0.1em}}} & Single Dose Intranasal Oxytocin Administration: Data from Healthy Younger and Older Adults (\cite{liu2022intranasal}) & CC0 & 85 & 85 & 0 & 0 \\
    \mbox{\href{https://openneuro.org/datasets/ds005503/versions/1.1.0}{\hspace{0.1em}\rule{0pt}{1.2em}SDRR\rule{0pt}{1.2em}\hspace{0.1em}}} & In silico discovery of representational relationships across visual cortex (\cite{gifford2024silico}) & CC0 & 6 & 12 & 0 & 0 \\
    \mbox{\href{https://openneuro.org/datasets/ds003826/versions/3.0.1}{\hspace{0.1em}\rule{0pt}{1.2em}SDSStudy\rule{0pt}{1.2em}\hspace{0.1em}}} & Structural (t1) images of 136 young healthy adults (\cite{zareba2022late}) & CC0 & 134 & 134 & 0 & 0 \\
    \mbox{\href{https://openneuro.org/datasets/ds003469/versions/1.0.0}{\hspace{0.1em}\rule{0pt}{1.2em}SDStudy\rule{0pt}{1.2em}\hspace{0.1em}}} & Speech disfluencies: Neurophysiological aspect in normal population  & CC0 & 81 & 81 & 0 & 0 \\
    \mbox{\href{https://openneuro.org/datasets/ds000222/versions/1.0.1}{\hspace{0.1em}\rule{0pt}{1.2em}SIVBM\rule{0pt}{1.2em}\hspace{0.1em}}} & https://openneuro.org/datasets/ds000222/versions/1.0.1 (\cite{fitzgerald2017sequential}) & CC0 & 79 & 79 & 0 & 0 \\
    \mbox{\href{https://openneuro.org/datasets/ds005266/versions/1.0.0}{\hspace{0.1em}\rule{0pt}{1.2em}SKIP\rule{0pt}{1.2em}\hspace{0.1em}}} & SoCal Kinesia and Incentivization for Parkinson's Disease (SKIP): Active Escape (\cite{dundon2024dissociation}) & CC0 & 68 & 68 & 0 & 0 \\
    \mbox{\href{https://openneuro.org/datasets/ds004920/versions/1.1.1}{\hspace{0.1em}\rule{0pt}{1.2em}SNRP\rule{0pt}{1.2em}\hspace{0.1em}}} & An fMRI dataset of social and nonsocial reward processing in young adults (\cite{smith2024fmri}) & CC0 & 58 & 58 & 0 & 0 \\
    \mbox{\href{https://openneuro.org/datasets/ds004889/versions/1.1.2}{\hspace{0.1em}\rule{0pt}{1.2em}SOOP\rule{0pt}{1.2em}\hspace{0.1em}}} & Stroke Outcome Optimization Project (SOOP) (\cite{absher2024stroke}) & CC0 & 1493 & 1493 & 0 & 1493 \\
    \mbox{\href{https://openneuro.org/datasets/ds005498/versions/1.1.1}{\hspace{0.1em}\rule{0pt}{1.2em}SPCC\rule{0pt}{1.2em}\hspace{0.1em}}} & Single-pulse TMS fMRI (\cite{glick2024concurrent}) & CC0 & 147 & 159 & 0 & 0 \\
    \mbox{\href{https://openneuro.org/datasets/ds004648/versions/1.0.0}{\hspace{0.1em}\rule{0pt}{1.2em}SQFC\rule{0pt}{1.2em}\hspace{0.1em}}} & Sympathovagal quotient and functional connectivity of control networks are related to gut Ruminococcaceae abundance in healthy men (\cite{miranda2024sympathovagal}) & CC0 & 86 & 86 & 0 & 0 \\
    \mbox{\href{https://openneuro.org/datasets/ds003745/versions/2.1.1}{\hspace{0.1em}\rule{0pt}{1.2em}SRPDM\rule{0pt}{1.2em}\hspace{0.1em}}} & An fMRI Dataset on Social Reward Processing and Decision Making in Younger and Older Adults (\cite{miranda2024sympathovagal}) & CC0 & 44 & 44 & 44 & 0 \\
    \mbox{\href{https://openneuro.org/datasets/ds005123/versions/1.1.3}{\hspace{0.1em}\rule{0pt}{1.2em}SRPStudy\rule{0pt}{1.2em}\hspace{0.1em}}} & Social Reward and Nonsocial Reward Processing Across the Adult Lifespan: An Interim Multi-echo fMRI and Diffusion Dataset (\cite{smith2024social}) & CC0 & 106 & 106 & 0 & 101 \\
    \mbox{\href{https://openneuro.org/datasets/ds000201/versions/1.0.3}{\hspace{0.1em}\rule{0pt}{1.2em}SSBStudy\rule{0pt}{1.2em}\hspace{0.1em}}} & The Stockholm Sleepy Brain Study: Effects of Sleep Deprivation on Cognitive and Emotional Processing in Young and Old (\cite{tamm2020combined}) & CC0 & 83 & 162 & 83 & 0 \\
    \mbox{\href{https://openneuro.org/datasets/ds004359/versions/1.0.0}{\hspace{0.1em}\rule{0pt}{1.2em}SSMD\rule{0pt}{1.2em}\hspace{0.1em}}} & SixthSense (\cite{zadbood2022neural}) & CC0 & 57 & 57 & 0 & 0 \\
    \mbox{\href{https://openneuro.org/datasets/ds003346/versions/1.1.2}{\hspace{0.1em}\rule{0pt}{1.2em}SUDMEX1\rule{0pt}{1.2em}\hspace{0.1em}}} & SUDMEX{\_}CONN: The Mexican dataset of cocaine use disorder patients. (\cite{garza2017effect}) & CC0 & 144 & 144 & 0 & 0 \\
    \mbox{\href{https://openneuro.org/datasets/ds003037/versions/2.1.0}{\hspace{0.1em}\rule{0pt}{1.2em}SUDMEX2\rule{0pt}{1.2em}\hspace{0.1em}}} & SUDMEX{\_}TMS (\cite{angeles2024mexican}) & CC0 & 53 & 155 & 0 & 0 \\
    \mbox{\href{https://openneuro.org/datasets/ds005600/versions/1.1.0}{\hspace{0.1em}\rule{0pt}{1.2em}TCHCC\rule{0pt}{1.2em}\hspace{0.1em}}} & Thalamocortical contributions to hierarchical cognitive control (\cite{chen2024thalamocortical}) & CC0 & 59 & 59 & 0 & 0 \\
    \mbox{\href{https://openneuro.org/datasets/ds004965/versions/1.0.1}{\hspace{0.1em}\rule{0pt}{1.2em}TCS\rule{0pt}{1.2em}\hspace{0.1em}}} & Truecrime (\cite{rominger2024mri}) & CC0 & 133 & 133 & 0 & 0 \\
    \mbox{\href{https://openneuro.org/datasets/ds005237/versions/1.0.6}{\hspace{0.1em}\rule{0pt}{1.2em}TDCP\rule{0pt}{1.2em}\hspace{0.1em}}} & Transdiagnostic Connectome Project (\cite{chopra2024transdiagnostic}) & CC0 & 238 & 238 & 238 & 0 \\
    \mbox{\href{https://openneuro.org/datasets/ds004086/versions/1.2.0}{\hspace{0.1em}\rule{0pt}{1.2em}TDRS\rule{0pt}{1.2em}\hspace{0.1em}}} & Triple Dissociation Revisited (\cite{van2022evidence}) & CC0 & 58 & 58 & 0 & 0 \\
    \mbox{\href{https://openneuro.org/datasets/ds000053/versions/00001}{\hspace{0.1em}\rule{0pt}{1.2em}TLAMNS\rule{0pt}{1.2em}\hspace{0.1em}}} & Training of loss aversion modulates neural sensitivity toward potential gains  & CC0 & 58 & 58 & 0 & 0 \\
    \mbox{\href{https://openneuro.org/datasets/ds004469/versions/1.1.4}{\hspace{0.1em}\rule{0pt}{1.2em}TLED\rule{0pt}{1.2em}\hspace{0.1em}}} & Temporal Lobe Epilepsy - UNAM (\cite{fajardo2024functional}) & CC0 & 66 & 66 & 0 & 0 \\
    \mbox{\href{https://openneuro.org/datasets/ds003138/versions/1.0.1}{\hspace{0.1em}\rule{0pt}{1.2em}TUWM\rule{0pt}{1.2em}\hspace{0.1em}}} & Tidying Up White Matter: Neuroplastic Transformations in Sensorimotor Tracts following Slackline Skill Acquisition (\cite{koschutnig2024tidying}) & CC0 & 53 & 159 & 0 & 0 \\
    \mbox{\href{https://openneuro.org/datasets/ds000030/versions/1.0.0}{\hspace{0.1em}\rule{0pt}{1.2em}UCLAC\rule{0pt}{1.2em}\hspace{0.1em}}} & UCLA Consortium for Neuropsychiatric Phenomics LA5c Study (\cite{gorgolewski2017preprocessed}) & CC0 & 262 & 262 & 0 & 0 \\
    \mbox{\href{https://openneuro.org/datasets/ds002717/versions/1.0.1}{\hspace{0.1em}\rule{0pt}{1.2em}URPL\rule{0pt}{1.2em}\hspace{0.1em}}} & Unexplained Repeated Pregnancy Loss is Associated with Altered Perceptual and Brain Responses to Men's Body-Odor (\cite{rozenkrantz2020unexplained}) & CC0 & 55 & 55 & 0 & 0 \\
    \mbox{\href{https://openneuro.org/datasets/ds005604/versions/1.0.1}{\hspace{0.1em}\rule{0pt}{1.2em}V4CStudy\rule{0pt}{1.2em}\hspace{0.1em}}} & V4 crowding  & CC0 & 49 & 55 & 42 & 0 \\
    \mbox{\href{https://openneuro.org/datasets/ds003717/versions/1.1.0}{\hspace{0.1em}\rule{0pt}{1.2em}VASP\rule{0pt}{1.2em}\hspace{0.1em}}} & Visual and audiovisual speech perception associated with increased functional connectivity between sensory and motor regions (\cite{peelle2022increased}) & CC0 & 60 & 60 & 60 & 0 \\
    \mbox{\href{https://openneuro.org/datasets/ds005449/versions/1.0.0}{\hspace{0.1em}\rule{0pt}{1.2em}VTIS\rule{0pt}{1.2em}\hspace{0.1em}}} & Valenced tactile information is evoked by neutral visual cues following emotional learning (\cite{ehlers2024valenced}) & CC0 & 21 & 21 & 0 & 0 \\
    \mbox{\href{https://openneuro.org/datasets/ds000115/versions/00001}{\hspace{0.1em}\rule{0pt}{1.2em}WMHCI\rule{0pt}{1.2em}\hspace{0.1em}}} & Working memory in healthy and schizophrenic individuals (\cite{repovvs2012working}) & CC0 & 99 & 99 & 0 & 0 \\
    \mbox{\href{https://openneuro.org/datasets/ds002424/versions/1.2.0}{\hspace{0.1em}\rule{0pt}{1.2em}WMRC\rule{0pt}{1.2em}\hspace{0.1em}}} & Working Memory and Reward in Children with and without Attention Deficit Hyperactivity Disorder (ADHD) (\cite{lytle2020neuroimaging}) & CC0 & 76 & 76 & 0 & 0 \\
    \mbox{\href{https://openneuro.org/datasets/ds000243/versions/00001}{\hspace{0.1em}\rule{0pt}{1.2em}WUStudy\rule{0pt}{1.2em}\hspace{0.1em}}} & Washington University 120 (\cite{power2013evidence}) & CC0 & 120 & 120 & 0 & 0 \\
\end{longtable}
\end{center}