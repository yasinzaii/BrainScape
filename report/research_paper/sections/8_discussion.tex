In this paper, we introduce BrainScape, an easy-to-use tool for the management, aggregation, 
and preprocessing of open-source anatomical MRI datasets.
BrainScape addresses known challenges associated with aggregating multimodal MRI datasets, 
including the fragmentation of specialized clinical data, incompatible dataset formats and organization, 
lack of automated data-handling workflows, and inconsistencies in demographic and clinical metadata.
BrainScape integrates heterogeneous MRI datasets while preserving each dataset's original structure, subject organization, and licensing details.
This design choice minimizes the risks of duplication, data loss, and inadvertent bias, enabling researchers to easily and reliably merge data from multiple sources 
for more comprehensive analyses.

A key motivation for pooling diverse datasets is to improve generalizability, 
bridging the gap between replicable findings and real-world clinical applicability 
(\cite{marek2024replicability, adkinson2024brain, yang2024limits}).
Although many large-scale consortia provide extensive datasets that support reproducibility, 
they often exhibit limited demographic representation and constrained dataset variability, 
reducing their broader generalizability. 
Such constraints can lead to shortcut learning in machine learning models, 
whereby models inadvertently capture patterns associated with confounding variables 
rather than true brain-behavior relationships (\cite{marek2024replicability, yang2024limits}). 
Another motivation behind BrainScape's design is the recognition that 
numerous specialized MRI datasets likely remain underutilized 
compared to the widely-used large-scale consortia datasets. 
Although these datasets often contain valuable information on rare phenotypes and specific clinical conditions, 
they are underutilized in pooling efforts due to complexities in data aggregation. 
By systematically integrating these diverse datasets, BrainScape promotes 
dataset heterogeneity, facilitating the development of robust, transferable models 
that accurately reflect the true variability observed in both healthy and clinical populations.

One of the strengths of BrainScape is its flexibility and scalability. 
The configurable and modular plugin-based architecture enables 
seamless integration of new data sources, imaging modalities, and preprocessing pipelines without requiring any modifications to the core codebase.
Researchers can add new multimodal MRI datasets by including new configuration files for those datasets.
Additionally, BrainScape provides the flexibility to include plugins for downloading data from other open-source MRI databases, 
as well as specialized preprocessing plugins for specific studies which may include custom preprocessing pipelines.
This design facilitates an environment of continuous improvement and customization.
We integrate \NumDatasets\ diverse datasets demonstrating the framework's scalability 
and capability to handle large-scale data aggregation.
Furthermore, BrainScape's ability to trace each file back to its source and maintain 
dataset-specific configurations demonstrates reproducibility by ensuring that the inclusion criteria, 
quality-control outcomes, and plugin parameters are recorded in a transparent and standardized manner.

By automating data downloading, mapping, validation, and preprocessing, BrainScape reduces the 
manual labor required for large-scale, multi-site data pooling. Researchers can therefore shift 
their focus away from tedious preparation tasks and concentrate on analysis, interpretation, and theory development. 
This is particularly helpful for deep learning and machine learning applications, where model 
performance benefits from the diversity and volume of training data (\cite{dishner2024survey}). Beyond aggregating diverse MRI scans, 
BrainScape also supports attaching demographic fields (such as age, sex, handedness, and clinical status), enabling 
researchers to build inclusive datasets that capture control and patient cohorts. 
This diversity is particularly crucial for rare phenotypes and targeted clinical cohorts, 
which may be overlooked in large-scale aggregated datasets yet offer critical insights into neurological function and pathology (\cite{thompson2014enigma}).

Despite its strengths, BrainScape has some limitations that also reflect wider issues in science.
First, there is an uneven coverage of demographic and clinical metadata across datasets, 
which may constrain the generalizability of analyses and introduce sampling biases. 
This fragmented demographic information likely arises because BrainScape aggregates resources from individual researchers 
with varying funding levels and research targets. 
As a result, we cannot expect the same level of annotations as found in multi-million-dollar projects 
such as HCP and UK Biobank (\cite{van2013wu, miller2016multimodal}).
Furthermore, the distribution of demographic variables does not fully reflect real-world populations. 
For instance, the current age distribution skews toward individuals between 5-30 years, with fewer participants above 30.
Similarly, most participants self-identify as White, while other racial or ethnic groups are comparatively underrepresented. 
Educational levels also exhibit an imbalance with more than half of the participants reporting high-level education (bachelor's, master's, or PhD).
Although these demographic imbalances mean the integrated dataset may not yet comprehensively represent 
the broader population, BrainScape's community-driven aggregation approach aims to progressively reduce 
these limitations by incorporating more diverse, heterogeneous datasets over time.
Potential reasons for the current bias include the nature of volunteer-based studies, 
recruitment from university populations, and missing demographic fields across datasets. 
Ongoing integration of additional datasets, particularly those emphasizing broader demographic 
and clinical diversity, will further enhance generalizability. 


Second, the distribution of MRI modalities is skewed, with a relatively high number of T1w images compared to T2w and FLAIR scans. 
This imbalance might limit the scope of studies that require multimodal MRI data for effective analysis, such as automated lesion segmentation (\cite{menze2014multimodal, spitzer2022interpretable}). 
Over time, community-driven contributions should address this imbalance by integrating additional datasets containing richer multimodal MRIs. 
Furthermore, licensing restrictions remain a concern; reliance on publicly available data means that the framework is subject to existing licensing restrictions 
and access barriers inherent in some proprietary datasets.
Although BrainScape tracks licensing terms and usage permissions, researchers must be aware of 
data use agreements and privacy regulations to ensure responsible collaboration. 
Additionally, downloading and preprocessing large-scale MRI data requires significant time and computational resources, 
potentially extending the overall study timeline and requiring access to high-performance computing resources.

Looking ahead, BrainScape can be extended to incorporate diffusion-weighted imaging (DWI), resting-state fMRI, 
and other specialized MRI sequences, enhancing the dataset capability to support multimodal analyses 
of structural and functional connectivity (\cite{van2013wu}). 
Furthermore, including an automated quality control module into the pipeline, such as machine-learning-based artifact detection 
may make it easier to include additional open-source datasets by reducing the need for manual quality control, which is labor-intensive and time-consuming. 
Such a module may also automatically detect poor-quality MRIs overlooked during manual review. 
Another key improvement is parallelizing the workflow; currently, BrainScape processes datasets sequentially, 
but enabling concurrent dataset handling could significantly reduce the overall processing time. 
Additionally, integrating new open-source brain datasets targeted at specialized analyses and rare phenotypes will further enrich the pooled dataset 
and expand the dataset utility for even more studies. 
Finally, BrainScape can be enhanced with new preprocessing plugins designed to implement specialized pipelines for targeted research studies.

In conclusion, our open-source framework addresses bottlenecks in MRI data aggregation and preprocessing by 
collecting datasets of diverse origins, automating workflows, and preserving original dataset integrity. 
It will serve as an effective tool for the neuroscience community by enabling the creation of more comprehensive, representative, and robust datasets.
While challenges such as inconsistent demographic data and licensing restrictions persist, 
BrainScape's flexible, transparent design allows for ongoing improvements through active community collaboration.
We believe that BrainScape contributes significantly to open science and enhances our understanding of brain structure and function.